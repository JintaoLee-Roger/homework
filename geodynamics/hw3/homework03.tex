\documentclass{article}

% \usepackage[UTF8, scheme = plain]{ctex}
\usepackage{fancyhdr}   % 页眉, 页脚, 页码设置
\usepackage{extramarks} % 
\usepackage{amsmath}    % 公式
% \usepackage{amsthm}     % 数学定理之类的,提供了 proof 等包 
% \usepackage{amsfonts}   % 数学字体
\usepackage{tikz}       % 绘图宏包 
% \usepackage[plain]{algorithm} % 制作算法图,伪代码
% \usepackage{algpseudocode}    % 伪代码
\usepackage{color}      % 颜色 (包括文字颜色等)
% \usepackage{listings}   % 代码, 使用xelatex
\usepackage{fontspec}   % 修改字体
\usepackage{float}      % 固定图片位置
\usepackage{indentfirst}      % 首行缩进

\setlength{\parindent}{2em}  %% 首行缩进
\setlength{\parskip}{1em} %% 设置段落间距
\linespread{1.1}    %% line spacing
\definecolor{ustcblue}{cmyk}{1,0.8,0,0}  %% ustc logo 颜色
\setmainfont{Times New Roman} %% 设置字体

% \newfontfamily\menlo{Menlo}
% % \newfontfamily\menlo{Consolas}
% \lstset{
%     columns=fixed,       
%     numbers=left,                                        % 在左侧显示行号
%     numberstyle=\tiny\color{gray},                       % 设定行号格式
%     frame=none,                                          % 不显示背景边框
%     backgroundcolor=\color[RGB]{245,245,244},            % 设定背景颜色
%     keywordstyle=\color[RGB]{40,40,255},                 % 设定关键字颜色
%     numberstyle=\footnotesize\color{darkgray},           
%     commentstyle=\it\color[RGB]{0,96,96},                % 设置代码注释的格式
%     stringstyle=\rmfamily\slshape\color[RGB]{128,0,0},   % 设置字符串格式
%     showstringspaces=true,                              % 不显示字符串中的空格
%     numberstyle=\small\menlo,
%     basicstyle=\small\menlo,
%     breaklines=true,
% }

%
% Basic Document Settings
%
%%% page layout
\topmargin=-0.45in
\evensidemargin=0in
\oddsidemargin=0in
\textwidth=6.5in
\textheight=9.0in
\headsep=0.25in

\pagestyle{fancy}
\lhead{\hmwkAuthorName}
\chead{\hmwkClass\ (\hmwkClassInstructor): \hmwkTitle}
\rhead{}
\lfoot{}
\cfoot{\thepage}

\renewcommand\headrulewidth{0.4pt}
\renewcommand\footrulewidth{0.4pt}

%
% Create Problem Sections
%

\newcommand{\enterProblemHeader}[1]{
    \nobreak\extramarks{}{Problem \arabic{#1} continued on next page\ldots}\nobreak{}
    \nobreak\extramarks{Problem \arabic{#1} (continued)}{Problem \arabic{#1} continued on next page\ldots}\nobreak{}
}

\newcommand{\exitProblemHeader}[1]{
    \nobreak\extramarks{Problem \arabic{#1} (continued)}{Problem \arabic{#1} continued on next page\ldots}\nobreak{}
    \stepcounter{#1}
    \nobreak\extramarks{Problem \arabic{#1}}{}\nobreak{}
}

\setcounter{secnumdepth}{0}
\newcounter{partCounter}
\newcounter{homeworkProblemCounter}
\setcounter{homeworkProblemCounter}{1}
\nobreak\extramarks{Problem \arabic{homeworkProblemCounter}}{}\nobreak{}

%
% Homework Problem Environment
%
% This environment takes an optional argument. When given, it will adjust the
% problem counter. This is useful for when the problems given for your
% assignment aren't sequential. See the last 3 problems of this template for an
% example.
%
\newenvironment{homeworkProblem}[1][-1]{
    \ifnum#1>0
        \setcounter{homeworkProblemCounter}{#1}
    \fi
    \subsection{Exercise \arabic{homeworkProblemCounter}}
    \setcounter{partCounter}{1}
    \enterProblemHeader{homeworkProblemCounter}
}{
    \exitProblemHeader{homeworkProblemCounter}
}


\newcommand{\hmwkTitle}{Homework\ \#3}
\newcommand{\hmwkDueDate}{\today}
\newcommand{\hmwkClass}{Geodynamics}
\newcommand{\hmwkClassInstructor}{Professor W. Len}
\newcommand{\hmwkAuthorName}{\textbf{Jintao Li}}
\newcommand{\hmwkAuthorID}{\textbf{SA20007037}}
\newcommand{\hmwkAuthoremail}{\textbf{E-mail: lijintao@mail.ustc.edu.cn}}


\renewcommand{\part}[1]{\textbf{ \\ (\alph{partCounter})  }\stepcounter{partCounter} }

% Alias for the Solution section header
\newcommand{\solution}{\textbf{\large \\ Solution: \\}}

%%%%
\newcommand{\mb}[1]{\mathbf{#1}}



\begin{document}

\begin{titlepage}
\begin{center}

\textcolor{ustcblue}{\includegraphics[width=0.4\textwidth]{../../inversion/ustc_logo_fig.pdf} \\ [1cm]}
% Title
{ \Huge \bfseries \hmwkClass\ \hmwkTitle}\\[1cm]

\large \textbf{\hmwkClassInstructor} \\ [5cm]

\large \hmwkAuthorName \\ [0.25cm]
\large \hmwkAuthorID \\ [0.25cm]
\large \hmwkAuthoremail
\vfill
% Bottom of the page
{\large \hmwkDueDate}

\end{center}
\end{titlepage}

\begin{center}
\section{Chapter 3: Elasticity and Flexure}
\end{center}


\begin{homeworkProblem}
The effect of large scale volcanic eruption on the crustal flexture.
\begin{figure}[H]
    \centering
    \includegraphics[width=6in, keepaspectratio]{fig01.pdf}
    \label{fig:fig01}
\end{figure}

Suppose volcanic eruption lasts for 2 $Ma$, forming
basalts accumulation with a height of 2 $km$. The
cross-section shape of the basalts are trapezoid, the
upper boundary of the trapezoid is 0.8 of the bottom
boundary. Basalts spread from the center at a speed of
300 $km/Ma$.
Some parameters: Young’s modulus 70 $GPa$, poisson's ratio 0.25, 
density of the basalts 2700 $kg/m^3$, crust density
2900 $kg/m^3$, elastic thickness of the crust 50 $km$.

Solve for the time variation of surface topography at
x=150, 300 and 450 $km$ from the eruption center.
Discuss the effects of different elastic thickness on the
results.


\solution

Refering to Brotchie's paper \textit{On Crustal Flexure}, the differential 
equation for deflection can be represented as:
\begin{equation}
    D \nabla^{4} w+\left(E T / R^{2}\right) w+\gamma w=q .
\end{equation}
And rewriting the formula in plane polar coordinate and the spherical coordinates of
the shell, it can be:
\begin{equation}
    \nabla^{4} w+\left(1 / l^{4}\right) w=q / D ,
    \label{eq:02}
\end{equation}
in which $l^{4} \equiv D /\left[\left(E T / R^{2}\right)+\gamma\right]$, $\omega$ 
is the radial displacementof the shell undernormalloadingof intensityq, D is
the flexural stiffness of the shell cross section 
$\equiv \left[E T^{3} / 12\left(1-v^{2}\right)\right]$, $T$ is the thickness of the
shell, $E$ is itsmodulusof elasticity, $\upsilon$ is Poisson's ratio for the 
shellmaterial, $R$ is the radius of its middle surface, $\gamma$ is the density 
of the enclosed liquid. 

We consider this the volcanic loading as \textit{variable loading}. 
And the deflections of crust for a volcanic eruption of variable thickness 
are found by superposition using uniform thickness solution. The variable 
thickness may be approximated by a stepped distribution. The sheet may then 
be considered to be composed of uniform layers of depth $h$ and radius $a_n$, 
and we choose step size $h = 5 m$. 

As to uniform loading, solving
the equation~\ref{eq:02}, we can obtain the solution:
\begin{equation}
    w_{i}=\frac{\gamma_{volcanic} h}{\gamma^{\prime}}
    \left(a \operatorname{ker}^{\prime} a \operatorname{ber} x - a 
    \operatorname{kei}^{\prime} a \operatorname{bei} x+1\right) ,
\end{equation}
and, outside the volcanic eruption, deflection $omega_0$ is:
\begin{equation}
    w_{0}=\frac{\gamma_{volcanic} h}{\gamma^{\prime}}
    (a \operatorname {ber}^{\prime} a \operatorname{ker} x
    - a \operatorname{bei}^{\prime} a \operatorname{kei} x) ,
\end{equation}
in which $h$ is uniform depth, $\gamma_{volcanic}$ is the density of basalts,
$\gamma$ is the density of mantle, and $\gamma^{\prime} = \gamma + ET/R^2$.

The result is illuminated in blow figure. The red, green and blue lines are 
represented the results of surface topography at x=150, 300 and 450 km from 
the eruption center, respectively. And the soild, dashed and dotted lines 
are the results of surface topography when crust thickness is 30, 50, 70 km, 
respectively.
\begin{figure}[H]
    \centering
    \includegraphics[width=6in, keepaspectratio]{result.pdf}
    \label{fig:result}
\end{figure}

As the inset shown, the deflection is lager when the position is closer to 
the eruption center. And the deflection tends to a stable value as the 
basalts accumulated except the farthest position, because the accumulation 
is not enough to bend the farthest position. The deflection declines fastest
at 0.5, 1.0, 1.5 Ma, respectively, and its reason is the basalt coverd this 
position at the time. 
In general, the thicker the crust, the less deflection there is, and 
the longer it takes to reach stability. 

\end{homeworkProblem}

\end{document}
