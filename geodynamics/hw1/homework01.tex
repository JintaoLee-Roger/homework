\documentclass{article}

\usepackage[UTF8, scheme = plain]{ctex}
\usepackage{fancyhdr}
\usepackage{extramarks}
\usepackage{amsmath}
\usepackage{amsthm}
\usepackage{amsfonts}
\usepackage{tikz}
\usepackage[plain]{algorithm}
\usepackage{algpseudocode}
\usepackage{color}
\usepackage{listings}
\usepackage{fontspec}
\usepackage{float}

\newfontfamily\menlo{Menlo}
% \newfontfamily\menlo{Consolas}
\lstset{
    columns=fixed,       
    numbers=left,                                        % 在左侧显示行号
    numberstyle=\tiny\color{gray},                       % 设定行号格式
    frame=none,                                          % 不显示背景边框
    backgroundcolor=\color[RGB]{245,245,244},            % 设定背景颜色
    keywordstyle=\color[RGB]{40,40,255},                 % 设定关键字颜色
    numberstyle=\footnotesize\color{darkgray},           
    commentstyle=\it\color[RGB]{0,96,96},                % 设置代码注释的格式
    stringstyle=\rmfamily\slshape\color[RGB]{128,0,0},   % 设置字符串格式
    showstringspaces=true,                              % 不显示字符串中的空格
    numberstyle=\small\menlo,
    basicstyle=\small\menlo,
    breaklines=true,
}

% \lstset{
%     language=Octave,                % the language of the code
%     basicstyle=\footnotesize,           % the size of the fonts that are used for the code
%     numbers=left,                   % where to put the line-numbers
%     numberstyle=\tiny\color{gray},  % the style that is used for the line-numbers
%     stepnumber=1,                   % the step between two line-numbers. If it's 1, each line 
%                                     % will be numbered
%     numbersep=5pt,                  % how far the line-numbers are from the code
%     backgroundcolor=\color{white},      % choose the background color. You must add \usepackage{color}
%     showspaces=false,               % show spaces adding particular underscores
%     showstringspaces=false,         % underline spaces within strings
%     showtabs=false,                 % show tabs within strings adding particular underscores
%     frame=single,                   % adds a frame around the code
%     rulecolor=\color{black},        % if not set, the frame-color may be changed on line-breaks within not-black text (e.g. commens (green here))
%     tabsize=2,                      % sets default tabsize to 2 spaces
%     captionpos=b,                   % sets the caption-position to bottom
%     breaklines=true,                % sets automatic line breaking
%     breakatwhitespace=false,        % sets if automatic breaks should only happen at whitespace
%     title=\lstname,                 % show the filename of files included with \lstinputlisting;
%                                     % also try caption instead of title
%     keywordstyle=\color{blue},          % keyword style
%     commentstyle=\color{dkgreen},       % comment style
%     stringstyle=\color{mauve},         % string literal style
%     escapeinside={\%*}{*)},            % if you want to add LaTeX within your code
%     morekeywords={*,...}               % if you want to add more keywords to the set
% }

\usetikzlibrary{automata,positioning}

%
% Basic Document Settings
%
%%% page layout
\topmargin=-0.45in
\evensidemargin=0in
\oddsidemargin=0in
\textwidth=6.5in
\textheight=9.0in
\headsep=0.25in

\linespread{1.1}    %%% line spacing

\definecolor{ustcblue}{cmyk}{1,0.8,0,0}

\pagestyle{fancy}
\lhead{\hmwkAuthorName}
\chead{\hmwkClass\ (\hmwkClassInstructor): \hmwkTitle}
\rhead{}
\lfoot{}
\cfoot{\thepage}

\renewcommand\headrulewidth{0.4pt}
\renewcommand\footrulewidth{0.4pt}

\setlength\parindent{0pt}

%
% Create Problem Sections
%

\newcommand{\enterProblemHeader}[1]{
    \nobreak\extramarks{}{Problem \arabic{#1} continued on next page\ldots}\nobreak{}
    \nobreak\extramarks{Problem \arabic{#1} (continued)}{Problem \arabic{#1} continued on next page\ldots}\nobreak{}
}

\newcommand{\exitProblemHeader}[1]{
    \nobreak\extramarks{Problem \arabic{#1} (continued)}{Problem \arabic{#1} continued on next page\ldots}\nobreak{}
    \stepcounter{#1}
    \nobreak\extramarks{Problem \arabic{#1}}{}\nobreak{}
}

\setcounter{secnumdepth}{0}
\newcounter{partCounter}
\newcounter{homeworkProblemCounter}
\setcounter{homeworkProblemCounter}{1}
\nobreak\extramarks{Problem \arabic{homeworkProblemCounter}}{}\nobreak{}

%
% Homework Problem Environment
%
% This environment takes an optional argument. When given, it will adjust the
% problem counter. This is useful for when the problems given for your
% assignment aren't sequential. See the last 3 problems of this template for an
% example.
%
\newenvironment{homeworkProblem}[1][-1]{
    \ifnum#1>0
        \setcounter{homeworkProblemCounter}{#1}
    \fi
    \subsection{Exercise \arabic{homeworkProblemCounter}}
    \setcounter{partCounter}{1}
    \enterProblemHeader{homeworkProblemCounter}
}{
    \exitProblemHeader{homeworkProblemCounter}
}

%
% Homework Details
%   - Title
%   - Due date
%   - Class
%   - Section/Time
%   - Instructor
%   - Author
%

\newcommand{\hmwkTitle}{Homework\ \#1}
\newcommand{\hmwkDueDate}{\today}
\newcommand{\hmwkClass}{Geodynamics}
\newcommand{\hmwkClassInstructor}{Professor W. Len}
\newcommand{\hmwkAuthorName}{\textbf{Jintao Li}}
\newcommand{\hmwkAuthorID}{\textbf{SA20007037}}
\newcommand{\hmwkAuthoremail}{\textbf{E-mail: lijintao@mail.ustc.edu.cn}}

%
% Title Page
%

% \title{
%     \vspace{2in}
%     \textbf{\hmwkClass:\ \hmwkTitle}\\
%     \normalsize\vspace{0.2in}\large{\hmwkDueDate}\\
%     \vspace{0.2in}\large{\textit{\hmwkClassInstructor}}
%     \vspace{3in}
% }

% \author{\hmwkAuthorName \\
% \hmwkAuthorID}
% \date{}

\renewcommand{\part}[1]{\textbf{ \\ (\alph{partCounter})  }\stepcounter{partCounter} }

%
% Various Helper Commands
%

% Useful for algorithms
\newcommand{\alg}[1]{\textsc{\bfseries \footnotesize #1}}

% For derivatives
\newcommand{\deriv}[1]{\frac{\mathrm{d}}{\mathrm{d}x} (#1)}

% For partial derivatives
\newcommand{\pderiv}[2]{\frac{\partial}{\partial #1} (#2)}

% Integral dx
\newcommand{\dx}{\mathrm{d}x}

% Alias for the Solution section header
\newcommand{\solution}{\textbf{\large \\ Solution: \\}}

% Probability commands: Expectation, Variance, Covariance, Bias
\newcommand{\E}{\mathrm{E}}
\newcommand{\Var}{\mathrm{Var}}
\newcommand{\Cov}{\mathrm{Cov}}
\newcommand{\Bias}{\mathrm{Bias}}

%
\newcommand{\mb}[1]{\mathbf{#1}}

\begin{document}

\begin{titlepage}

\begin{center}

\textcolor{ustcblue}{\includegraphics[width=0.4\textwidth]{./ustc_logo_fig.pdf} \\ [1cm]}
% Title
{ \Huge \bfseries \hmwkClass\ \hmwkTitle}\\[1cm]

\large \textbf{\hmwkClassInstructor} \\ [5cm]

\large \hmwkAuthorName \\ [0.25cm]
\large \hmwkAuthorID \\ [0.25cm]
\large \hmwkAuthoremail
\vfill
% Bottom of the page
{\large March 24, 2021}

\end{center}

\end{titlepage}

\begin{center}
\section{Chapter 1: Plate Tectonics}
\end{center}

\begin{homeworkProblem}[1]
Download ETOPO1 topography data, plot globe and regional topography map,
using a good color scale.

\solution

GMT (version = 6.1.1) has included topography data, so we can use it directly
without downloading from the website: https://www.ngdc.noaa.gov/mgg/global/global.html.
But if you want to download the data from offical website, you can do it.

\textbf{bash code}
\begin{lstlisting}[language={bash}]
#!/bin/sh
# set font 
gmt set FONT_ANNOT_PRIMARY=10
gmt set FONT_LABEL=14

# globe, using the default cpt (i.e. geo), which is the most beautiful colormap.
gmt begin globe_relief pdf
gmt grdimage @earth_relief_03m -JH180/15c -I+d
gmt colorbar -DJBC+w10c/0.25c+o0.0c/0.3c -Bxa3f+l"Topography/km" -C+Ukm
gmt end show

# regional, Philippines, Lon: 115-130 E, Lan: 0-22 N
gmt begin phlippines_relif pdf
gmt grdimage @earth_relief_15s -R115/130/0/22 -JM15c -I+d -Ba -BNWes
gmt colorbar -DJBC+w10c/0.25c+o0.0c/0.3c -Bxa3f+l"Topography/km" -C+Ukm
gmt end show

# if using the topography data downloaded from offical website, uncommit follow code
# gmt begin globe_relief_offical pdf
# gmt grdimage ETOPO1_Bed_g_gdal.grd -JH180/15c -I+d -Cgeo
# gmt colorbar -DJBC+w10c/0.25c+o0.0c/0.3c -Bxa3f+l"Topography/km" -C+Ukm
# gmt end show
\end{lstlisting}

% This code can also be found in the attachment "draw_topo.sh".

\textbf{The figures}
\begin{figure}[H]
    \centering
    \includegraphics[width=6in, keepaspectratio]{globe_relief.pdf}
    \label{fig:globe_relief}
\end{figure}

\begin{figure}[H]
    \centering
    \includegraphics[width=6in, keepaspectratio]{phlippines_relif.pdf}
    \label{fig:phlippines_relif}
\end{figure}

\end{homeworkProblem}

\pagebreak


\begin{homeworkProblem}[2]
Download IRIS earthquake data, (for certain time period), plot globe and 
regional earthquake distribution, including location, magnitude, depth, etc. 
For those who are interested, also plot source mechanism for typical earthquake (
i.e. beach ball).

\solution

We can download source mechanism data from the website: https://www.globalcmt.org/CMTsearch.html.
Due to the large number of earthquakes, we choose globe earthquakes with Mw $\geq$ 6.0,
date from 2011/01/01 to 2021/01/01 and earthquakes in  Phlippines with 
Mw $\geq$ 6.0, date from 2011/01/01 to 2021/01/01.

\textbf{bash code}
\begin{lstlisting}[language={bash}]
#!/bin/sh
gmt set FONT_ANNOT_PRIMARY=10
gmt set FONT_LABEL=12

# globe 
gmt begin earthquake_globe pdf
gmt grdimage @earth_relief_03m -I+d -Ba -BNWes
# generate depth cpt file
echo 0 purple@30 70 purple@30 > depth.cpt
echo 70 green@30 300 green@30 >> depth.cpt
echo 300 red@30 800 red@30 >> depth.cpt
gmt meca cmt -Sm0.3c -Zdepth.cpt 
# draw legend (in the bottom)
( 
cat << EOF 
H 10,0 Earthquake Location
L 6,0 C (2011/01/01-2021/01/01)
G 0.1c
L 8,0 C Magnitude(Mw)
G 0.1c
N 3
S 0.1i c 0.11i - black 0.2i 6
S 0.1i c 0.12i - black 0.2i 7
S 0.1i c 0.13i - black 0.2i 8
N 1
G 0.1c
L 8,0 C Depth(km)
G 0.1c
N 3
L 7,0 R @;purple;0-70km@;;
L 7,0 R @;green;70-300km@;;
L 7,0 R @;red;300-800km@;;
G 1.0c
B globe 0.3i 0.08i+ml -Bxa3f+l"Topo(km)" -C+Ukm --FONT_ANNOT_PRIMARY=8p --MAP_FRAME_WIDTH=1p --FONT_LABEL=10p
EOF
) > tmp
gmt legend tmp -F+gazure1@10 -DjBC+w8c+l1.2+o0.0c/-5c -C0.1i/0.1i
echo 123 33.05 43 3.62 -0.44 -3.18 0.90 2.46 -1.35 24 0 0 > tmp
echo 126.6 33.05 171 -0.71 -0.26 0.96 0.44 0.81 -0.07 24 0 0 >> tmp
echo 130.5 33.05 302 0.34 0.16 -0.50 -0.77 -4.57 -1.58 24 0 0 >> tmp    
gmt meca tmp -Sm0.3c -Zdepth.cpt
gmt end show

# regional (Phlippines)
gmt begin earthquake_phlippins pdf
gmt grdimage @earth_relief_15s -R115/130/0/22 -JM15c -I+d -Ba -BNWes
# plot beach ball
gmt meca phili -Sm0.3c -Zdepth.cpt -JM15c 
# plot survey line (the third question needs this)
echo 117 6 A > tmp
echo 129 6 B >> tmp
gmt plot tmp -W1p,black,-.-
gmt text tmp -F+f10p -D0c/-0.2c 
# draw legend (Right Top)
( 
cat << EOF 
H 10,0 Earthquake Location
L 6,0 C (2011/01/01-2021/01/01)
G 0.1c
L 8,0 C Magnitude(Mw)
G 0.1c
N 4
S 0.1i c 0.10i - black 0.2i 5
S 0.1i c 0.11i - black 0.2i 6
S 0.1i c 0.12i - black 0.2i 7
S 0.1i c 0.13i - black 0.2i 8
N 1
G 0.1c
L 8,0 C Depth(km)
G 0.1c
N 3
L 7,0 R @;purple;0-70km@;;
L 7,0 R @;green;70-300km@;;
L 7,0 R @;red;300-800km@;;
N 1
G 0.1c
L 8,0 C Survey Line
G 0.1c
N 2
L 7,0 R A(117 6)
L 7,0 R B(129 6)
EOF
) > tmp
gmt legend tmp -F+gazure1@10 -DjTR+w4c+l1.2 -C0.1i/0.1i
echo 123 33.05 43 3.62 -0.44 -3.18 0.90 2.46 -1.35 24 0 0 > tmp
echo 126.6 33.05 171 -0.71 -0.26 0.96 0.44 0.81 -0.07 24 0 0 >> tmp
echo 130.5 33.05 302 0.34 0.16 -0.50 -0.77 -4.57 -1.58 24 0 0 >> tmp    
gmt meca tmp -Sm0.3c -Zdepth.cpt
gmt colorbar -DJBC+w10c/0.25c+o0.0c/0.3c -Bxa3f+l"Topography/km" -C+Ukm
gmt end show
\end{lstlisting}

% This code can also be found in the attachment "draw_earthquake.sh".

\textbf{The figure}
\begin{figure}[H]
    \centering
    \includegraphics[width=6in, keepaspectratio]{earthquake_globe.pdf}
    \label{fig:earthquake_globe}
\end{figure}

\begin{figure}[H]
    \centering
    \includegraphics[width=6in, keepaspectratio]{earthquake_phlippins.pdf}
    \label{fig:earthquake_phlippins}
\end{figure}

    
\end{homeworkProblem}

\pagebreak

\begin{homeworkProblem}[3]
Choose certain cross-section of a subducting plate, plot the Benioff zone with 
earthquake distribution (down to 660 km depth).

\solution

We choos a cross-section from A(117E, 6N) to B(128E, 6N) (see Figure~\ref{fig:earthquake_phlippins}).
\textbf{bash code}
\begin{lstlisting}[language={bash}]
#!/bin/sh
gmt set FONT_ANNOT_PRIMARY=10
gmt set FONT_LABEL=12

gmt begin section pdf
gmt basemap -R0/12/-8000/2000 -JX10c/3c -BWrtb -Bya2000f+l"Elevation (m)"
# generate depth cpt file
echo 0 purple@30 70 purple@30 > depth.cpt
echo 70 green@30 300 green@30 >> depth.cpt
echo 300 red@30 700 red@30 >> depth.cpt

# commit A and B
echo 0 2200 A > tmp
echo 12 2200 B >> tmp
gmt text tmp -F+f10p+jBC -N -D0c/0.1c

# project
gmt project -C117/6 -E129/6 -G0.1 | gmt grdtrack -G@earth_relief_03m > tmp
echo 0 0 > tmp2 
echo 12 0 >> tmp2
gmt plot tmp2 -Wblack -Glightblue -L+y-8000
gmt plot tmp -i2,3 -Wblack -Ggray -L+y-8000

# benioff zone
gmt basemap -R0/12/0/700 -JX10c/-5c -Bya200f100+l"Focal depth (km)" -Bxa2f1+l"Distance"+u"\260" -BWSrt -Y-5.5c
echo 5.5 480 Benioff zone | gmt text -F+f14p,5,blue=solid+a50+jBL
gmt coupe phili -Q -L -Sm0.3c -Aa117/6/129/6/90/300/0/700f -Zdepth.cpt
( 
cat << EOF 
H 8,4 Events (Mw >= 5)
S 0.1i c 0.20 purple 0.1p,black 0.20i 0-70 km
S 0.1i c 0.20 green 0.1p,black 0.20i 70-300 km
S 0.1i c 0.20 red 0.1p,black 0.20i 300-700 km
EOF
) > tmp
gmt legend tmp -DjBR+w1i+l1.5+o0.05i/0.04i -F+g255+p0.25p
gmt end show

rm *.cpt tmp*
\end{lstlisting}

% This code can also be found in the attachment "draw_section.sh".

\textbf{The figure}
\begin{figure}[H]
    \centering
    \includegraphics[width=6in, keepaspectratio]{section.pdf}
    \label{fig:section}
\end{figure}


\end{homeworkProblem}


\end{document}
