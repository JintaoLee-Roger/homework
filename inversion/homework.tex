\documentclass{article}

\usepackage{fancyhdr}
\usepackage{extramarks}
\usepackage{amsmath}
\usepackage{amsthm}
\usepackage{amsfonts}
\usepackage{tikz}
\usepackage[plain]{algorithm}
\usepackage{algpseudocode}
\usepackage{color}

\usetikzlibrary{automata,positioning}

%
% Basic Document Settings
%
%%% page layout
\topmargin=-0.45in
\evensidemargin=0in
\oddsidemargin=0in
\textwidth=6.5in
\textheight=9.0in
\headsep=0.25in

\linespread{1.1}    %%% line spacing

\definecolor{ustcblue}{cmyk}{1,0.8,0,0}

\pagestyle{fancy}
\lhead{\hmwkAuthorName}
\chead{\hmwkClass\ (\hmwkClassInstructor): \hmwkTitle}
\rhead{}
\lfoot{}
\cfoot{\thepage}

\renewcommand\headrulewidth{0.4pt}
\renewcommand\footrulewidth{0.4pt}

\setlength\parindent{0pt}

%
% Create Problem Sections
%

\newcommand{\enterProblemHeader}[1]{
    \nobreak\extramarks{}{Problem \arabic{#1} continued on next page\ldots}\nobreak{}
    \nobreak\extramarks{Problem \arabic{#1} (continued)}{Problem \arabic{#1} continued on next page\ldots}\nobreak{}
}

\newcommand{\exitProblemHeader}[1]{
    \nobreak\extramarks{Problem \arabic{#1} (continued)}{Problem \arabic{#1} continued on next page\ldots}\nobreak{}
    \stepcounter{#1}
    \nobreak\extramarks{Problem \arabic{#1}}{}\nobreak{}
}

\setcounter{secnumdepth}{0}
\newcounter{partCounter}
\newcounter{homeworkProblemCounter}
\setcounter{homeworkProblemCounter}{1}
\nobreak\extramarks{Problem \arabic{homeworkProblemCounter}}{}\nobreak{}

%
% Homework Problem Environment
%
% This environment takes an optional argument. When given, it will adjust the
% problem counter. This is useful for when the problems given for your
% assignment aren't sequential. See the last 3 problems of this template for an
% example.
%
\newenvironment{homeworkProblem}[1][-1]{
    \ifnum#1>0
        \setcounter{homeworkProblemCounter}{#1}
    \fi
    \subsection{Exercise \arabic{homeworkProblemCounter}}
    \setcounter{partCounter}{1}
    \enterProblemHeader{homeworkProblemCounter}
}{
    \exitProblemHeader{homeworkProblemCounter}
}

%
% Homework Details
%   - Title
%   - Due date
%   - Class
%   - Section/Time
%   - Instructor
%   - Author
%

\newcommand{\hmwkTitle}{Homework\ \#1}
\newcommand{\hmwkDueDate}{\today}
\newcommand{\hmwkClass}{Inversion}
\newcommand{\hmwkClassInstructor}{Professor H. Yao \& H. Zhang}
\newcommand{\hmwkAuthorName}{\textbf{Jintao Li}}
\newcommand{\hmwkAuthorID}{\textbf{SA20007037}}
\newcommand{\hmwkAuthoremail}{\textbf{E-mail: lijintao@mail.ustc.edu.cn}}

%
% Title Page
%

% \title{
%     \vspace{2in}
%     \textbf{\hmwkClass:\ \hmwkTitle}\\
%     \normalsize\vspace{0.2in}\large{\hmwkDueDate}\\
%     \vspace{0.2in}\large{\textit{\hmwkClassInstructor}}
%     \vspace{3in}
% }

% \author{\hmwkAuthorName \\
% \hmwkAuthorID}
% \date{}

\renewcommand{\part}[1]{\textbf{(\alph{partCounter})\ }\stepcounter{partCounter}}

%
% Various Helper Commands
%

% Useful for algorithms
\newcommand{\alg}[1]{\textsc{\bfseries \footnotesize #1}}

% For derivatives
\newcommand{\deriv}[1]{\frac{\mathrm{d}}{\mathrm{d}x} (#1)}

% For partial derivatives
\newcommand{\pderiv}[2]{\frac{\partial}{\partial #1} (#2)}

% Integral dx
\newcommand{\dx}{\mathrm{d}x}

% Alias for the Solution section header
\newcommand{\solution}{\textbf{\large \\ Solution: \\}}

% Probability commands: Expectation, Variance, Covariance, Bias
\newcommand{\E}{\mathrm{E}}
\newcommand{\Var}{\mathrm{Var}}
\newcommand{\Cov}{\mathrm{Cov}}
\newcommand{\Bias}{\mathrm{Bias}}

\begin{document}

\begin{titlepage}

\begin{center}

\textcolor{ustcblue}{\includegraphics[width=0.25\textwidth]{./ustc_logo_fig.pdf} \\ [1cm]}
% Title
{ \Huge \bfseries \hmwkClass\ \hmwkTitle}\\[1cm]

\large \textbf{\hmwkClassInstructor} \\ [5cm]

\large \hmwkAuthorName \\ [0.25cm]
\large \hmwkAuthorID \\ [0.25cm]
\large \hmwkAuthoremail
\vfill
% Bottom of the page
{\large \today}

\end{center}

\end{titlepage}

\begin{center}
\section{Appendix A}
\end{center}

\begin{homeworkProblem}[4]
Let 
\begin{equation}
\mathbf{A} = 
\left[
\begin{matrix}
    1 & 2 & 3 & 4 \\
    2 & 2 & 1 & 3 \\
    4 & 6 & 7 & 11
\end{matrix}
\right]
\end{equation}
Find bases for $\mathcal{N}(\mathbf{A})$, $\mathcal{R}(\mathbf{A})$, $\mathcal{N}(\mathbf{A}^T)$, 
and $\mathcal{R}(\mathbf{A}^T)$. What are the dimensions of the four subspaces?

\solution

To find $\mathcal{N}(\mathbf{A})$, we solve the system of equations $\mathbf{Ax} = 0$,
\begin{equation}
    \mathbf{Ax} = \left[\begin{matrix}
        1 & 2 & 3 & 4 \\
        2 & 2 & 1 & 3 \\
        4 & 6 & 7 & 11
    \end{matrix}\right]\left[\begin{matrix}
        x_1 \\ x_2 \\ x_3 \\ x_4
    \end{matrix}\right] = \left[\begin{matrix}
        0 \\ 0 \\ 0
    \end{matrix}\right]
\end{equation}
Put the system of equations into an augmented matrix,
\begin{equation}
\left[\begin{array}{llll|l}
1 & 2 & 3 & 4 & 0 \\
2 & 2 & 1 & 3 & 0 \\
4 & 6 & 7 & 11 & 0
\end{array}\right]
\end{equation}
and then find the reduced row echelon form (\textbf{RREF}), 
\begin{equation}
\left[\begin{array}{llll|l}
1 & 0 & -2 & -1 & 0 \\
0 & 1 & \frac{5}{2} & \frac{5}{2} & 0 \\
0 & 0 & 0 & 0 & 0
\end{array}\right] 
\end{equation}
We can find that 
\begin{equation}
    \mathbf{x} = \left[ \begin{matrix} x_1 \\ x_2 \\ x_3 \\ x_4 \end{matrix} \right] = 
    \left[ \begin{matrix} {2x_3 + x_4} \\ {-\frac{5}{2}x_3 - \frac{5}{2}x_4} \\ x_3 \\ x_4 \end{matrix} \right] = 
    \left[ \begin{matrix} 2 \\ -\frac{5}{2} \\ 1 \\ 0 \end{matrix} \right] x_3 + 
    \left[ \begin{matrix} 1 \\ -\frac{5}{2} \\ 0 \\ 1 \end{matrix} \right] x_4
\end{equation}

So, 
\begin{equation}
    \mathcal{N}(\mathbf{A}) = \mathbf{space}\left(
        \left[\begin{matrix} 2 \\ - \frac{5}{2} \\ 1 \\ 0 \end{matrix}\right]
        \left[\begin{matrix} 1 \\ - \frac{5}{2} \\ 0 \\ 1
        \end{matrix}\right]\right) ,
\end{equation}
and the dimension of $\mathcal{N}(\mathbf{A})$ is \textbf{2}. \\

To find $\mathcal{R}(\mathbf{A})$, the equations becomes $\mathbf{Ax} = \mathbf{b}$,
\begin{equation}
    \mathbf{Ax} = \left[\begin{matrix}
        1 & 2 & 3 & 4 \\
        2 & 2 & 1 & 3 \\
        4 & 6 & 7 & 11
    \end{matrix}\right]\left[\begin{matrix}
        x_1 \\ x_2 \\ x_3 \\ x_4
    \end{matrix}\right] = \left[\begin{matrix}
        b_1 \\ b_2 \\ b_3
    \end{matrix}\right] = \mathbf{b} 
\end{equation}

Because the \textbf{RREF} of \textbf{A} is 
\begin{equation}
    \left[\begin{matrix}
        1 & 0 & -2 & -1 \\
        0 & 1 & \frac{5}{2} & \frac{5}{2} \\
        0 & 0 & 0 & 0
    \end{matrix}\right]
\end{equation}

We can write $\mathbf{b}$ as a linear combination of the first two columns of $\mathbf{A}$:
\begin{equation}
    \mathbf{b} = x_1 \left[ \begin{matrix}
        1 \\ 2 \\ 4 
    \end{matrix} \right] + x_2 \left[ \begin{matrix}
        2 \\ 2 \\ 6
    \end{matrix} \right] , 
\end{equation}

i.e. 
\begin{equation}
    \mathcal{R}(\mathbf{A}) = \mathbf{space} \left(
        \left[\begin{matrix} 1 \\ 2 \\ 4 \end{matrix}\right]
        \left[\begin{matrix} 2 \\ 2 \\ 6 \end{matrix}\right]\right), 
\end{equation}

and the dimension of $\mathcal{R}(\mathbf{A})$ is \textbf{2}. \\

To find $\mathcal{N}(\mathbf{A}^T)$ and $\mathcal{R}(\mathbf{A}^T)$, we first calculate 
the \textbf{RREF} of $\mathbf{A}^T$.
\begin{equation}
    \mathbf{A}^T = \left[\begin{matrix}
        1 & 2 & 4 \\
        2 & 2 & 6 \\
        3 & 1 & 7 \\
        4 & 3 & 11
    \end{matrix}\right] ,
\end{equation}

\begin{equation}
    \mathbf{RREF\ of\ A\ is\ } 
    \left[\begin{matrix}
        1 & 0 & 2 \\
        0 & 1 & 1 \\
        0 & 0 & 0 \\
        0 & 0 & 0
    \end{matrix}\right] .
\end{equation}

So, 
\begin{equation}
    \mathbf{x} = \left[\begin{matrix}
        -2 \\ -1 \\ 0
    \end{matrix}\right] x_3 ,
\end{equation}

i.e.
\begin{equation}
    \begin{aligned}
    \mathcal{N}(\mathbf{A}^T) = \mathbf{space}\left(\left[\begin{matrix}
        -2 \\ -1 \\ 0
    \end{matrix}\right]\right) , \\ 
    \mathcal{R}(\mathbf{A}^T) = \mathbf{space}\left(\left[\begin{matrix}
        1 \\ 2 \\ 3 \\ 4
    \end{matrix}\right] \left[\begin{matrix}
        2 \\ 2 \\ 1 \\ 3
    \end{matrix}\right]\right) .
    \end{aligned}
\end{equation}
And the dimension of $\mathcal{N}(\mathbf{A}^T)$ is \textbf{1}, the dimension of 
$\mathcal{R}(\mathbf{A}^T)$ is \textbf{2}.

\end{homeworkProblem}

\begin{homeworkProblem}[10]
Show that if $\mathbf{x} \perp \mathbf{y}$, then 
\begin{equation}
\left\| \mathbf{x} + \mathbf{y} \right\|^2_2 = 
\left\| \mathbf{x} \right\|^2_2 +
\left\| \mathbf{y} \right\|^2_2 .
\end{equation}

\solution

\end{homeworkProblem}

\begin{homeworkProblem}[11]
In this exercise, we will derive the formula (A.88) for the 1-norm of a matrix.
Begin with the optimization problem 
\begin{equation}
\left\| \mathbf{A} \right\|_1 = 
\max_{\| \mathbf{x} \|_1 = 1} \| \mathbf{Ax}\|_1 .
\end{equation}

\part

Show that if $\|\mathbf{x}\|_1 = 1$, then
\begin{equation}
\|\mathbf{Ax}\|_1 \leq 
\max_{j} \sum\limits_{i=1}^m \left| A_{i, j} \right| .
\end{equation}

\solution

\part

Find a vector $\mathbf{x}$ such that $\|\mathbf{x}\|_1 = 1$, and 
\begin{equation}
\|\mathbf{Ax}\|_1 = \max_{j} \sum\limits_{i=1}^m \left| A_{i, j} \right| .
\end{equation}

\solution

\part

Conclude that 
\begin{equation}
\|\mathbf{A}\|_1 = \max_{\|\mathbf{x}\|_1 = 1} \|\mathbf{Ax}\|_1 = 
\max_{j} \sum\limits_{i=1}^m \left| A_{i, j} \right| .
\end{equation}

\solution

\end{homeworkProblem}

\begin{center}
    \section{Appendix B}
\end{center}

\begin{homeworkProblem}[6]
Suppose that $\mathbf{X}=(X_1, X_2)^T$ is a vector composed of two random variables with a 
multivariate normal distribution with expected value $\mu$ and covariance matrix $\mathbf{C}$,
and that $\mathbf{A}$ is a 2 by 2 matrix. Use properties of expected value and covariance to 
show that $\mathbf{y} = \mathbf{Ax}$ has expected value $\mathbf{A}\mu$ and covariance $\mathbf{ACA}^T$

\solution

\end{homeworkProblem}

\begin{homeworkProblem}[9]
Using MATLAB, repeat the following experiment 1000 times. Generate five exponentially distributed 
random numbers from the exponential probability density function (B.10) with means $\mu = 1/\lambda = 10$. 
You may find the library function \textbf{exprand} to be useful here. Use (B.74) to calculate a 95\% 
confidence interval for the 1000 mean determinations. How many times out of the 1000 experiments did 
the 95\% confidence interval cover the expected value of 10? What happens if you instead generate 50 
exponentially distributed random numbers at time? Discuss your results. 

\solution

\end{homeworkProblem}

\begin{center}
    \section{Appendix C}
\end{center}

\begin{homeworkProblem}[1]
Let 
\begin{equation}
    f(\mathbf{x}) = x_1^2 x_2^2 - 2x_1 x_2^2 + x_2^2 - 3x_1^2 x_2 + 
    12x_1 x_2 - 12x_2 + 6 .
\end{equation}
Find the gradient, $\nabla f(\mathbf{x})$, and Hessian, $\mathbf{H}(f(\mathbf{x}))$. 
What are the critical points of $f$? Which of these are minima and maxima of $f$?

\solution

\end{homeworkProblem}

\begin{homeworkProblem}[2]
Find a Taylor's series approximation for $f(\mathbf{x} + \Delta\mathbf{x})$, where
\begin{equation}
    f(\mathbf{x}) = e^{-(x_1 + x_2)^2}
\end{equation}
is near the point 
\begin{equation}
    \mathbf{x} = \left[ 
        \begin{matrix}
        2 \\
        3
        \end{matrix} \right] .
\end{equation}

\solution

\end{homeworkProblem}

\end{document}
